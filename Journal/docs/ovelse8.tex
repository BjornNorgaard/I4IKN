\section{Øvelse 8 - TCP/IP socket programming}

\subsection{Øvelsesbeskrivelse}

\begin{enumerate}
	\item Skriv en iterativ TCP-server med support for en client ad gangen, som kan modtage	en tekststreng fra en client. Serveren skal køre i en virtuel Linux-maskine.
	Tekststrengen skal indeholde et filnavn + en eventuel stiangivelse. Tilsammen skal informationen i tekststrengen udpege en fil af en vilkårlig type/størrelse beliggende i
	serveren, som en tilsluttet client ønsker at hente fra serveren. Hvis filen ikke findes	skal serveren returnere en fejlmelding til client’en. Hvis filen findes skal den overføres
	fra server til client i segmenter på 1000 bytes ad gangen indtil filen er overført fuldstændigt. Serverens portnummer skal være 9000.
	\item Skriv en TCP-client kørende i en anden virtuel Linux-maskine, som kan sende en tekststreng, indtastet af operatøren. Tekststrengen skal indeholde et filnavn + en
	eventuel stiangivelse til en fil i TCP-serveren, som er beskrevet i punkt 1. Client’en skal modtage den ønskede fil fejlfrit fra serveren – eller udskrive en fejlmelding hvis
	filen ikke findes i serveren. Client-applikationen skal kunne startes fra en terminal med kommandoen:
	\item Som kvalitetskontrol for client/server systemet skal den overførte fil kunne sammenlignes med den oprindelige fil vha. terminal-kommandoen:
	cmp <afsendt fil> < modtaget fil> 
	eller
	diff –s <afsendt fil> < modtaget fil>
	<afsendt fil> er overført til client vha. af email, ftp eller anden pålidelig, ikke	proprietær overføringsmetode.
	Der må ikke være forskel mellem filerne, hverken mht. til størrelse eller mht. indhold.
\end{enumerate}

